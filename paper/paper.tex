\documentclass[10pt,conference]{IEEEtran}
\IEEEoverridecommandlockouts
% The preceding line is only needed to identify funding in the first footnote. If that is unneeded, please comment it out.
\usepackage{cite}
\usepackage{amsmath,amssymb,amsfonts}
\usepackage{algorithmic}
\usepackage{graphicx}
\usepackage{textcomp}
\usepackage{xcolor}
\usepackage{subfig}
\def\BibTeX{{\rm B\kern-.05em{\sc i\kern-.025em b}\kern-.08em
    T\kern-.1667em\lower.7ex\hbox{E}\kern-.125emX}}
    \def\code#1{\texttt{#1}}
\begin{document}

\title{A Second Look at the Dynamics of the JavaScript Package Ecosystem\\}

\author{\IEEEauthorblockN{Kevin de Haan, Gregory Neagu, Frederic Sauve-Hoover, Abram Hindle}
\IEEEauthorblockA{\textit{Department of Computing Science} \\
\textit{University of Alberta}\\
Edmonton, Canada\\
Email: \{kdehaan,neagu,rsauveho,abram.hindle\}@ualberta.ca}
}


\maketitle

% Footnote should maybe be a full source, or just one link
\begin{abstract}
In recent years, the tools and packages most commonly involved with JavaScript development have evolved rapidly.
Newer packages such as Angular and React have experienced a marked increase in popularity among developers, while frameworks such as jQuery
have begun to phase out.\footnote[1]{\label{adoption}https://insights.stackoverflow.com/survey/2016\#technology-most-popular-technologies, 
https://insights.stackoverflow.com/survey/2017\#technology-\_-frameworks-libraries-and-other-technologies, https://insights.stackoverflow.com/survey/2018\#technology-\_-frameworks-libraries-and-tools}
For this reason we take a second look at a 2016 paper by Wittern, Suter and Rajagopalan \cite{Wittern:2016} to see what aspects of the JavaScript package ecosystem have changed, 
and if previously observed trends have remained constant.
In the original paper the authors use the \emph{node package manager} (\code{npm}) to gain 
insight into the JavaScript ecosystem as a whole, and data from projects publicly hosted on \code{GitHub} to observe an alternative measure of popularity. We adhere to the same methods of analysis, and extend the data to capture
more recent information up to April 1\textsuperscript{st} 2019.
Ultimately, this second look aims to discover if recent years have had any significant effects on ecosystem-wide trends, and provide developers with further insight into how packages are used and evolve.
\end{abstract}

\section{Introduction}
This paper is a replication of \emph{A Look at the Dynamics of the JavaScript Package Ecosystem}\cite{Wittern:2016} that performs extensive analysis of 
the \emph{node package manager} (\code{npm}), a popular distributor of JavaScript-based packages. Since the publishing of the original paper, the usage and scale of \code{npm} has only grown. With more than three times as many hosted packages (now over 750,000) 
and over ten times as many weekly package downloads (now over ten billion per week), the raw volume of data and the complexity of package dependency graphs has increased significantly. 
Additionally, the major frameworks used in JavaScript development have undergone a rapid transformation as packages such as Angular and React are adopted\footnotemark[\ref{adoption}].
The core contributions we make are as follows:
\begin{itemize}
  \item We replicate and verify the results found in the original paper for the window of October 1\textsuperscript{st} 2010 to September 1\textsuperscript{st} 2015.
  \item We extend the analysis to the time period of September 2\textsuperscript{nd} 2015 to April 1\textsuperscript{st} 2019, and evaluate whether patterns and trends noted in the original paper are still observable.
  \item We investigate whether the continued evolution of the JavaScript package ecosystem has affected the relationships between various measures of package popularity.
  \item We determine if the ongoing maturation of the JavaScript ecosystem has resulted in tangible changes to version numbering or adoption practices.
\end{itemize}

% \section{Related Work}


\section{Methodology}

\subsection{Data Collection}


\section{Results and Discussion}


\subsection{Threats to Validity}



\section{Conclusion}


\begin{thebibliography}{00}
  \bibitem{Wittern:2016}
  Erik Wittern, Philippe Suter, and Shriram Rajagopalan.
  \newblock A look at the dynamics of the javascript package ecosystem.
  \newblock In {\em Proceedings of the 13th International Conference on Mining
    Software Repositories}, MSR '16, pages 351--361, New York, NY, USA, 2016.
    ACM.



\end{thebibliography}
\vspace{12pt}

\end{document}
