\documentclass[10pt,conference]{IEEEtran}
\IEEEoverridecommandlockouts
% The preceding line is only needed to identify funding in the first footnote. If that is unneeded, please comment it out.
\usepackage{cite}
\usepackage{amsmath,amssymb,amsfonts}
\usepackage{algorithmic}
\usepackage{graphicx}
\usepackage{textcomp}
\usepackage{xcolor}
\usepackage{subfig}
\def\BibTeX{{\rm B\kern-.05em{\sc i\kern-.025em b}\kern-.08em
    T\kern-.1667em\lower.7ex\hbox{E}\kern-.125emX}}
    \def\code#1{\texttt{#1}}
\begin{document}

\title{A Second Look at the Dynamics of the JavaScript Package Ecosystem\\}

\author{\IEEEauthorblockN{ Kevin de Haan, Gregory Neagu, Frederic Sauve-Hoover, Abram Hindle}
\IEEEauthorblockA{\textit{Department of Computing Science} \\
\textit{University of Alberta}\\
Edmonton, Canada\\
Email: \{kdehaan,neagu,rsauveho,abram.hindle\}@ualberta.ca}
}


\maketitle

% Footnote should maybe be a full source, or just one link
\begin{abstract}
In recent years, the tools and packages most commonly involved with JavaScript development have evolved rapidly.
Newer packages such as Angular and React have experienced a marked increase in popularity among developers, while libraries such as jQuery
have begun to phase out.\footnote[1]{https://insights.stackoverflow.com/survey/2016\#technology-most-popular-technologies, 
https://insights.stackoverflow.com/survey/2017\#technology-\_-frameworks-libraries-and-other-technologies, https://insights.stackoverflow.com/survey/2018\#technology-\_-frameworks-libraries-and-tools}
For this reason, we take a second look at a paper by Wittern, Suter and Rajagopalan \cite{Wittern:2016} based on data up to September 2015 to see what has changed, and if previously observed trends have remained constant.
In the original paper, the authors use the \emph{node package manager} (\code{npm}) to gain 
insight into the JavaScript ecosystem as a whole. \code{npm}, a hosting service for JavaScript-based software, has only grown in popularity since
the original paper, with more than three times as many hosted packages (now over 750,000) and over ten times as many weekly package downloads (now over ten billion per week).
Additionally, data collected from projects publicly hosted on \code{GitHub} allow us to observe an alternative measure of popularity. 
Ultimately, this second look aims to discover if recent years have had any significant effects on ecosystem-wide trends, and provide developers with further insight into how packages are used and evolve.
\end{abstract}

\section{Introduction}


\subsection{Research Questions}
\begin{itemize}
  \item 

\end{itemize}



\section{Related Work}


\section{Methodology}


\subsection{Data Collection}

\section{Results and Discussion}


\subsection{Threats to Validity}



\section{Conclusion}


\begin{thebibliography}{00}
  \bibitem{Wittern:2016}
  Erik Wittern, Philippe Suter, and Shriram Rajagopalan.
  \newblock A look at the dynamics of the javascript package ecosystem.
  \newblock In {\em Proceedings of the 13th International Conference on Mining
    Software Repositories}, MSR '16, pages 351--361, New York, NY, USA, 2016.
    ACM.



\end{thebibliography}
\vspace{12pt}

\end{document}
